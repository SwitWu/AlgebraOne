\section{Introduction}


Historically algebra was created to understand zeros of
polynomial equations. Along the way the importance of various
system of numbers and their symmetries became evident.
The importance of symmetries of objects in other parts of
math and other sciences turned it into an important part of
algebra which has connections with geometry, analysis,
combinatorics, topology, etc. Symmetries of objects are studied
in group theory, which is the main subject of our course.

We start by recalling some of the basic concepts from
set theory and congruence arithmetic.

\paragraph{Equivalence Relation}
Let $X$ be a non-empty set. A relation over $X$ is a subset
$\R$ of $X\times X$. If $(x,y)\in\R$, we say $x$ is $\R$-related
to $y$ and write $x\R y$. Suppose $\R$ is a relation over $X$.
Then:
\begin{itemize}
    \item $\R$ is called reflexive if $\forall x\in X$, $x\R x$.
    \item $\R$ is called symmetric if $\forall x,y\in X$, $x\R y\Rightarrow y\R x$.
    \item $\R$ is called transitive if $\forall x,y\in X$, $x\R y,y\R z\Rightarrow x\R z$.
\end{itemize}

$\R$ is called an equivalent relation if $\R$ is reflexive, symmetric
and transitive.

Equivalent relations are essentially about equality with
respect to certain measurment. The following example
is an important indication of this concept:

\begin{example}
    Suppose $X$ and $Y$ are two non-empty sets and $f:X\to Y$
    is a function. Let $\sim$ be the following relation over $X$:
    \[\forall x_1,x_2\in X,x_1\sim x_2\Longleftrightarrow f(x_1)=f(x_2).\]
    Then $\sim$ is an equivalent relation.
\end{example}

Alternatively equivalent relations partition $X$ into subsets
that share the same property. For instance in the previous example
the shared property is having the same value under the function $f$.
Let's recall that $P$ is called a partition of a non-empty set
$X$, if 
\begin{itemize}
    \item $P$ consists of non-empty subsets of $X$, (subsets)
    \item $A,B\in P$ and $A\neq B\Rightarrow A\cap B=\varnothing$, (disjoint)
    \item $\forall x\in X$, $\exists A\in P$ such that $x\in A$. (covering)
\end{itemize}

Suppose $P$ is a partition of $X$. Then we get a classification function
from $X$ to $P$: $X\to P$, $x\mapsto [x]_P$ where $[x]_P$ is the 
unique element of $P$ which contains $x$. Notice that because of
the covering condition $x$ is contained in some element of $P$,
and because of the disjointness condition $x$ is in a unique
element of $P$. By the previous example, $x\sim y\Leftrightarrow [x]_P=[y]_P$
is an equivalent relation. So we can obtain the following lemma:

\begin{lemma}
    Suppose $P$ is a partition of a non-empty set $X$.
    For $x,y\in X$, $x\sim y$ if $x$ and $y$ are in the same element
    of $P$. Then $\sim$ is an equivalent relation.
\end{lemma}

\begin{proof}
    For $x\in X$, let $[x]_P$ be the unique element of $P$
    which contains $x$. So $x\mapsto[x]_P$ is a function from $X$
    to $P$. By the previous example, $x\sim y\Leftrightarrow[x]_P=[y]_P$
    is an equivalent relation over $X$. Notice that this means 
    $x\sim y$ exactly when $x$ and $y$ are in the same element of $P$.
\end{proof}

Starting with an equivalent relation $\sim$ over a non-empty
set $X$, we can partition $X$ with respect to $\sim$,
as we show next.

For $x\in X$, let $[x]:=\{y\in X\mid y\sim x\}$ (all the elements
that are $\sim$-related to $x$) We call $[x]$ the equivalent class
of $x$ with respect to $\sim$. When $x\sim y$, we say $x$ is equivalent to
$y$ with respect to $\sim$.

\begin{proposition}
    Suppose $\sim$ is an equivalent relation over a non-empty
    set $X$. Then $\{[x]\mid x\in X\}$ is a partition of $X$.
\end{proposition}

\begin{lemma}
    $x\sim y\Leftrightarrow [x]=[y]$.
\end{lemma}

\begin{proof}[Proof of lemma]
    ($\Leftarrow$) We want to show $[x]=[y]\Rightarrow x\sim y$.
    $x\sim x\Rightarrow x\in[x]\Rightarrow x\in[y]\Rightarrow x\sim y$.

    ($\Rightarrow$) $x\sim y\Rightarrow[x]=[y]$.
    To show equality of sets $[x]$ and $[y]$, it is necessary
    and sufficient to prove $[x]\subset [y]$ and $[y]\subset [x]$.
    Let's start by proving $[x]\subset [y]$.
    $\forall z\in[x]$, $z\sim x$. Combining with $x\sim y$
    we have $z\sim y$, so $z\in[y]$. Hence $[x]\subset [y]$.
    This means we showed
    \begin{equation}
        x\sim y\Rightarrow [x]\subset [y].\tag{I}
    \end{equation}
    Notice that $x\sim y\Rightarrow y\sim x$. Therefore,
    by (I), $[y]\subset [x]$.
    \begin{equation}
        x\sim y\Rightarrow y\sim x\Rightarrow [y]=[x].\tag{II}
    \end{equation}
    By (I) and (II), we have $x\sim y\Rightarrow [x]=[y]$.
\end{proof}


\begin{proof}[Proof of Proposition]
    $\forall x\in X$, $x\sim x$. Thus $x\in[x]$.
    This implies that $[x]$s are non-empty subsets and they cover
    $X$. Next we show the disjointness property.
    Suppose $z\in[x]\cap[y]$.
    \[\left.\begin{array}[]{c}
        z\in[x]\Rightarrow z\sim x\Rightarrow [z]=[x]\\
        z\in[y]\Rightarrow z\sim y\Rightarrow [z]=[y]
    \end{array}\right\}\Rightarrow [x]=[y].\]
    We showed that $[x]\cap[y]\neq\varnothing\Rightarrow [x]=[y]$.
    The contrapositive of this statement is 
    \[[x]\neq [y]\Rightarrow[x]\cap[y]=\varnothing,\]
    which is the disjointness property.
\end{proof}


Next we recall the congruence modulo $n$ relation which 
plays an important role in our course.