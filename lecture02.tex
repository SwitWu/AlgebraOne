\section{Congruences}


The set of integers is denoted by $\FZ$. For $a,b\in\FZ$,
we say $a$ divide $b$ and write $a\mid b$ if $b=ak$ for some
integer $k$. Suppose $n$ is a non-zero integer, we say
$a$ is congruent to $b$ modulo $n$ and write $a\equiv b\pmod n$
or $a\modnequiv b$ if $n\mid a-b$.

\begin{lemma}
    $\modnequiv$ is an equivalent relation over $\FZ$.
\end{lemma}

As we have seen eariler, every equivalent relation
gives us a partition and an equality function.
For $a\in\FZ$, the equivalent class of $a$ with respect
to $\modnequiv$ is called the mod-$n$ residue class of $a$
and it is denoted by $[a]_n$. By the results that we 
proved for equivalent relations we have that
\begin{itemize}
    \item $\{[a]_n\mid a\in\FZ\}$ is a partiton of $\FZ$
    \item $a\modnequiv b\Leftrightarrow [a]_n=[b]_n$.
\end{itemize}

The partition $\{[a]_n\mid a\in\FZ\}$ is denoted by $\FZ_n$
and it is called the set of integers modulo $n$.
Notice that
\[\begin{split}
    b\in[a]_n
    &\Longleftrightarrow b\modnequiv a\\
    &\Longleftrightarrow n\mid b-a\\
    &\Longleftrightarrow \exists k\in\FZ,b-a=nk\\
    &\Longleftrightarrow \exists k\in\FZ,b=a+nk\\
    &\Longleftrightarrow b\in\{a+nk\mid k\in\FZ\}
\end{split}\]

To understand the set $\FZ_n$ better,
we recall the well-ordering principle and
the long division. One of the important properties
of positive integers is the well-ordering principle.

\paragraph{The well-ordering principle}
Every non-empty subset of the set $\FZ^{\geq 0}$
of non-negative integers has a minimum.

Using the well-ordering principle we can prove the
division algorithm.

\begin{theorem}[The division algorithm]
    For every $a\in\FZ$, $b\in\FZ\backslash\{0\}$,
    there is a unique pair $(q,r)$ of integers
    such that $a=bq+r$ and $0\leq r<|b|$.
\end{theorem}

Suppose the pair $(q,r)$ is given in the long division
algorithm. Then $q$ is called the quotient of $a$ divided by 
$b$ and $r$ is called the remainder of $a$ divided by $b$.

Using the long division algorithm we obtain that $\FZ_n$
has $n$ elements.

\begin{proposition}
    Suppose $n$ is an integer more than $1$. Then 
    \[\FZ_n=\{[0]_n,[1]_n,\dots,[n-1]_n\}\text{\quad and\quad}|\FZ_n|=n.\]
\end{proposition}

\begin{proof}
    For every $a\in\FZ$, by the long division algorithm,
    there are integers $q,r$ such that $a=nq+r$ and $0\leq r<n$.
    $a=nq+r$ implies that $a\modnequiv r$. Hence $[a]_n=[r]_n$.
    Thus $[a]_n\in\{[0]_n,[1]_n,\dots,[n-1]_n\}$. Next we show $[i]_n\neq [j]_n$
    if $0\leq j<i\leq n-1$. Suppose to the contrary $[i]_n=[j]_n$.
    \[\begin{split}
        [i]_n=[j]_n&\Rightarrow i\modnequiv j\Rightarrow n\mid i-j \\
                   &\Rightarrow i-j=nq\text{\quad for some integer }q
    \end{split}\]
    Notice that $0\leq j<i<n$ implies $0<i-j<n$, therefore $0<nq<n$.
    Then $0<q<1$ which is a contradiction as there is no integer more than
    $0$ and less than $1$. This completes the proof.
\end{proof}

The set of integers modulo $n$ have arithmetic operations similar to the set
of integers. This can be viewed as a generalization of the fact that
\begin{table}
    \centering
    \begin{tabular}{c|cc}
        $+$ & even & odd \\
        \hline
        even & even & odd \\
        odd & odd & even
    \end{tabular}
    \quad
    \begin{tabular}{c|cc}
        $\cdot$ & even & odd \\
        \hline
        even & even & even \\
        odd & even & odd
    \end{tabular}
\end{table}

No matter what even or odd numbers we pick the above
tables hold. So we can simply view the above tables as
operations for the set of even numbers and the set of odd numbers.

\begin{lemma}
    The following are well-defined operations on $\FZ_n$:
    \[[a]_n+[b]_n=[a+b]_n\text{\quad and\quad}[a]_n\cdot [b]_n=[a\cdot b]_n.\]
\end{lemma}

Before we go to the proof of this lemma, let's try to understand 
what it says. Notice that $[a]_n$ is a set and $a\in[a]_n$.
We say $a$ is a representative of this residue class.
Recall that $[a]_n$ is the equivalence class of $a$
with respect to the equivalent relation $\modnequiv$.
From all the elements of $\FZ$ that are in the same class
we are choosing a representative. We are doing the same for
the residue class $[b]_n$. Then we are adding the chosen representatives
in $\FZ$, and next we are considering the residue class of the
sum (or product) of these representatives.
A priori it is not clear why these operations do not depend
on the choice of the representatives. This is an extremely
important process. We will be using the same idea later when we define
factor groups. Whenever you are using a representative from a 
class and applying certain logic or operations to obtain
a claim for the entire class, you have to be extra careful.
You have to make sure that you are not ``stareotyping'' and
your process is independent of the choice of a representative.

\begin{proof}
    Suppose $[a]_n=[a']_n$ and $[b]_n=[b']_n$. We have to show that
    \[[a+b]_n=[a'+b'_n]\text{\quad and\quad}[a\cdot b]_n=[a'\cdot b']_n.\]
    \begin{equation}
        [a]_n=[a']_n\Rightarrow a\modnequiv a'\Rightarrow\exists k\in\FZ,a-a'=nk\tag{I}
    \end{equation}
    \begin{equation}
        [b]_n=[b']_n\Rightarrow b\modnequiv b'\Rightarrow\exists l\in\FZ,b-b'=nl\tag{II}
    \end{equation}
    Therefore,
    \[\begin{split}
        (I)+(II)
        &\Rightarrow (a-a')+(b-b')=nk+nl=n(k+l)\\
        &\Rightarrow (a+b)-(a'+b')=n(k+l)\\
        &\Rightarrow a+b\modnequiv a'+b'\\
        &\Rightarrow [a+b]_n=[a'+b']_n.
    \end{split}\]

    Next we want to show $[a\cdot b]_n=[a'\cdot b']_n$. Notice
    that 
    \[\begin{split}
        [a\cdot b]_n=[a'\cdot b']_n
        &\Leftrightarrow a\cdot b\modnequiv a'\cdot b'\\
        &\Leftrightarrow a\cdot b-a'\cdot b'\text{ is a multiple of }n
    \end{split}\]
    We change one factor at a time. This is similar to how we show
    the product rule in caculus.
    \[\begin{split}
        a\cdot b-a'\cdot b'
        &=a\cdot b-a'\cdot b+a'\cdot b-a'\cdot b'\\
        &=(a-a')\cdot b+a'\cdot (b-b')\\
        &=(nk)\cdot b+a'\cdot(nl)\\
        &=n\cdot (kb+a'l).
    \end{split}\]
    Therefore $a\cdot b\modnequiv a'\cdot b'$, so $[a\cdot b]_n=[a'\cdot b']_n$.
\end{proof}

Next we see that the above operations on $\FZ_n$
satisfy associativity and distribution. Moreover
$\FZ_n$ has neutral elements with respect to both $+$ and $\cdot$.
Every element has an additive inverse. Later we will see
what elements have multiplicative inverse.

\begin{proposition}
    For every $[a]_n$, $[b]_n$, $[c]_n\in\FZ_n$
    we have 
    \begin{description}
        \item[Associative] \[[a]_n+([b]_n+[c]_n)=([a]_n+[b]_n)+[c]_n\] \[[a]_n\cdot([b]_n\cdot[c]_n)=([a]_n\cdot [b]_n)\cdot[a]_n\]
        \item[Neutral element] \[[a]_n+[0]_n=[0]_n+[a]_n=[a]_n\] \[[a]_n\cdot [1]_n=[1]_n\cdot [a]_n=[a]_n\]
        \item[Additive inverse]  \[[a]_n+[-a]_n=[-a]_n+[a]_n=[0]_n\]
        \item[Commutative] \[[a]_n+[b]_n=[b]_n+[a]_n\] \[[a]_n\cdot [b]_n=[b]_n\cdot [a]_n\]
        \item[Distributive] \[[a]_n\cdot([b]_n+[c]_n)=[a]_n\cdot [b]_n+[a]_n\cdot [c]_n\] \[([b]_n+[c]_n)\cdot [a]_n=[b]_n\cdot [a]_n+[c]_n\cdot [a]_n\] 
    \end{description} 
\end{proposition}
All the claims are straightforward conclusions of similar
properties for integers.