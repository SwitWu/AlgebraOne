\section{Multiplicative structure of integers mod $n$}
Here we want to investigate what elements of $\mathbb{Z}_n$
have multiplicative inverse.
\begin{definition}
    We say $[a]_n\in\mathbb{Z}_n$ has a multiplicative inverse
    if $[a]_n\cdot [a']_n=[1]_n$ for some $[a']_n\in\mathbb{Z}_n$. We say $[a]_n$
    is a unit of $\mathbb{Z}_n$ if it has a multiplicative 
    inverse. The set of all the units of $\mathbb{Z}_n$ is denoted
    by $\mathbb{Z}_n^*$.
\end{definition}

\begin{theorem}
    Suppose $n\in\mathbb{Z}$ and $n\geq 2$. Then
    \[\mathbb{Z}_n^*=\{[a]_n\mid \gcd(a,n)=1\}.\]
    Moreover $|\mathbb{Z}_n^*|=|\{a\in\mathbb{Z}\mid 1\leq a\leq n,\gcd(a,n)=1\}|$.
\end{theorem}

(The left hand side of the above equality is denoted
by $\phi(n)$ and it is called \textbf{Euler's phi function}.)

\begin{proof}
    ($\subset$) Suppose $[a]_n\in\mathbb{Z}_n^*$. Then $[a]_n[a']_n=[1]_n$
    for some $a'\in\mathbb{Z}$. Hence $[aa']_n=[1]_n$ which implies
    that $aa'\equiv 1\pmod n$. (Earlier we proved that $b\equiv b'\pmod n$
    implies $\gcd(b,n)=\gcd(b',n)$). Hence $\gcd(aa',n)=\gcd(1,n)=1$.
    Therefore $\gcd(a,n)=1$.

    ($\supset$) Suppose $\gcd(a,n)=1$. Then $1=ra+sn$
    for some $r,s\in\mathbb{Z}$. Since $ra+sn=1$, we obtain that
    \[ra\equiv 1\pmod n.\]
    This implies that $[ra]_n=[1]_n$, and so 
    $[r]_n[a]_n=[1]_n$. Therefore $[a]_n\in\mathbb{Z}_n$.
\end{proof}

\begin{example}
    List all the elements of $\FZ_6^*$
\end{example}

\begin{solve}
    $\FZ_6^*=\{[a]_6\mid 1\leq a\leq 6,\gcd(a,6)=1\}=\{[1]_6,[5]_6\}$.
\end{solve}

\begin{example}
    List all the elements of $\FZ_8^*$.
\end{example}

\begin{proof}
    $\FZ_8^*=\{[1]_8,[3]_8,[5]_8,[7]_8\}$.
\end{proof}

\begin{proposition}
    Suppose $p$ is prime. Then $\FZ_p^*=\FZ_p\backslash\{[0]_p\}$.
\end{proposition}

\begin{proof}
    By the previous theorem,
    \[\FZ_p^*=\{[a]_p\mid 1\leq a\leq p,\gcd(a,p)=1\}.\]
    Since $p$ is prime, for every integer $1\leq a<p$ we have
    $\gcd(a,p)=1$. Hence $\FZ_p^*=\{[a]_p\mid 1\leq a<p\}$.
    Since $\FZ_p=\{[0]_p,[1]_p,\dots,[p-1]_p\}$,
    we obtain that $\FZ_p^*=\FZ_p\backslash\{[0]_p\}$.
\end{proof}

The converse of the previous proposition is essentially true:

\begin{proposition}
    Suppose $n\in\FZ$, $n\geq 2$. If $\FZ_n^*=\FZ_n\backslash\{[0]_n\}$,
    then $n$ is prime.
\end{proposition}

\begin{proof}
    If $\FZ_n^*=\FZ_n\backslash\{[0]_n\}$, then $\phi(n)=n-1$.
    This means
    \[|\{a\in\FZ\mid 1\leq a\leq n,\gcd(a,n)=1\}|=n-1.\]
    So $n$ does not have any divisor in the interval $(1,\dots,n)$.
    Since $n\geq 2$, we deduce that $n$ is prime.
\end{proof}

\begin{example}
    Suppose $p$ is prime and $k\in\FZ$. Then $\phi(p^k)=p^k-p^{k-1}$.
\end{example}

\begin{solve}
    We show that $\gcd(a,p^k)=1\Leftrightarrow p\nmid a$.

    ($\Rightarrow$) We show the contrapositive. If $p\mid a$,
    then $p$ is a common divisor of $a$ and $p^k$, and so $\gcd(a,p^k)\neq 1$.

    ($\Leftarrow$) We proceed by induction on $k$.

    Base case, $k=1$.

    Since $p\nmid a$, $\gcd(a,p)\neq p$. Since $p$ has exactly two positive
    divisors $1$ and $p$, we deduce that $\gcd(a,p)=1$.

    Induction step. $\gcd(a,p^k)=1\Rightarrow\gcd(a,p^{k+1})=1$.

    By the base case, $\gcd(a,p)=1$. Then $\gcd(d,p)=1$ where
    $d=\gcd(a,p^{k+1})$. Since $d\mid p^{k+1}$ and $\gcd(d,p)=1$,
    by Euclid's lemma, $d\mid p^k$. So $d$ is a common divisor of $a$
    and $p^k$. Hence $d\leq\gcd(a,p^k)$. By the induction hypothesis
    $\gcd(a,p^k)=1$, and so $d=1$ (Notice that $d\geq 1$). This means
    $\gcd(a,p^{k+1})=1$, and claim follows.

    By the above claim,
    \[\begin{split}
        \phi(p^k)
        &=|\{a\in\FZ\mid 1\leq a\leq p^k,\gcd(a,p^k)=1\}|\\
        &=|\{a\in\FZ\mid 1\leq a\leq p^k,p\nmid a\}|\\
        &=|[1\dots p^k]\backslash\{a\in\FZ\mid 1\leq a\leq p^k,p\mid a\}|\\
        &=p^k-|\{a\in\FZ\mid 1\leq a\leq p^k,p\mid a\}|.
    \end{split}\]
    \[\begin{split}
        1\leq a\leq p^k,p\mid a
        &\Longleftrightarrow a=pa'\text{ and }1\leq pa'\leq p^k\\
        &\Longleftrightarrow a=pa'\text{ and }1\leq a'\leq p^{k-1}
    \end{split}\]
    So there are $p^{k-1}$ many $a'$s that satisfy $1\leq a\leq p^k$ and $p\mid a$.
    Hence \[\phi(p^k)=p^k-p^{k-1}.\qedhere\]
\end{solve}

Next we show that $\FZ_n^*$ is closed under multiplication.
This type of property plays an important role in group theory.

\begin{theorem}
    Suppose $n\in\FZ$ and $n\geq 2$. Then 

    (Operator) For every $[a]_n,[b]_n\in\FZ_n^*$, $[a]_n\cdot [b]_n\in\FZ_n^*$.

    (Associative) For every $[a]_n,[b]_n,[c]_n\in\FZ_n^*$,
    \[([a]_n\cdot [b]_n)\cdot [c]_n=[a]_n\cdot([b]_n\cdot [c]_n).\]

    (Neutral element) For every $[a]_n\in\FZ_n^*$, $[a]_n\cdot[1]_n=[1]_n\cdot[a]_n=[a]_n$.

    (Inverse) For every $[a]_n\in\FZ_n^*$, there is $[a']_n\in\FZ_n^*$ such that
    \[[a]_n\cdot [a']_n=[a']_n\cdot [a]_n=[1]_n.\]
\end{theorem}

\begin{proof}
    We have already proved that multiplication in $\FZ_n$
    is associative and $[1]_n$ is a neutral element of multiplication.
    Next we show that $\FZ_n^*$ is closed under multiplication.
    Suppose $[a]_n,[b]_n\in\FZ_n^*$. Then there are 
    $[a']_n,[b']_n\in\FZ_n$ such that $[a]_n\cdot [a']_n=[1]_n$
    and $[b]_n\cdot [b']_n=[1]_n$. Hence
    \[\begin{split}
        ([a]_n\cdot [b]_n)([b']_n\cdot [a']_n)
        &=[a]_n([b]_n\cdot [b']_n)[a']_n\\
        &=([a]_n\cdot [1]_n)[a']_n\\
        &=[a]_n\cdot [a']_n=[1]_n.
    \end{split}\]
    This means $[a]_n\cdot [b]_n\in\FZ_n^*$. Finally let's us discuss
    why every element of $\FZ_n^*$ has an inverse in $\FZ_n^*$.

    Since $[a]_n\in\FZ_n^*$, there is $[a']_n$ in $\FZ_n$
    such that 
    \[[a]_n\cdot [a']_n=[1]_n.\]
    The above equality implies that $[a']_n\cdot [a]_n=[1]_n$,
    and so $[a']_n\in\FZ_n^*$. This completes the proof.
\end{proof}