\section{Group}
Group theory is (mostly) about symmetrices of objects.
In some interesting examples in geometry, combinatorics,
or even chemistry, knowing the symmetrices uniquely 
determine the object. One can say that at a meta-level,
the whole mathematics (and in general sciences) is about
finding patterns as we want to reduce the amount of
data that we need to store. (Lowering the complexity
of the objects that we are studying.)

We start with an axiomatic definition of groups, and then 
give the relation with symmetrices.
\begin{definition}
    Suppose $G$ is a non-empty set and $(g_1,g_2)\mapsto g_1\cdot g_2$
    is an operator on $G$ (that means it is a function from $G\times G$
    to $G$). We say $(G,\cdot)$ (or simply $G$) is a group if the 
    following properties hold.
    \begin{itemize}
        \item (Associative) $\forall g_1,g_2,g_3\in G$, $g_1\cdot (g_2\cdot g_3)=(g_1\cdot g_2)\cdot g_3$
        \item (Neutral element) $\exists e\in G$, $\forall g\in G$, $g\cdot e=e\cdot g=g$
        \item (Inverse) $\forall g\in G$, $\exists g'\in G$, $g\cdot g'=g'\cdot g=e$, where $e$ is a neutral element.
    \end{itemize}
\end{definition}

We have already seen some exmaples of groups.
\begin{example}
    $(\mathbb{Z},+)$, $(\mathbb{Q},+)$, $(\FR,+)$ and $(\mathbb{C},+)$ are groups.
\end{example}

\begin{example}
    $(\mathbb{Q}\backslash\{0\},\cdot)$, $(\FR\backslash\{0\},\cdot)$, and 
    $(\mathbb{C}\backslash\{0\},\cdot)$ are groups.
\end{example}

\begin{example}
    For every integer $n\geq 2$, $(\mathbb{Z}_n,+)$ is a group.
\end{example}

\begin{example}
    For every integer $n\geq 2$, $(\mathbb{Z}_n^*,\cdot)$ is a group.
\end{example}

In some of the examples, we showed the uniqueness
of a neutral element when it exists. Next we show
this property in a general setting.

\begin{lemma}
    Suppose $G$ is a non-empty set, and $(g_1,g_2)\mapsto g_1\cdot g_2$
    is an operation. Suppose $e,e'\in G$ are neutral
    elements of $\cdot$. Then $e=e'$. In particular,
    in a group, there is a unique neutral element.
\end{lemma}

\begin{proof}
    Since $e$ is a neutral element, $e\cdot e'=e'$.
    Because $e'$ is a neutral element, $e\cdot e'=e$.
    Altogether we have 
    \[e'=e\cdot e'=e'.\qedhere\]
\end{proof}

Next we show the uniqueness of inverse in a group.

\begin{lemma}
    Suppose $(G,\cdot)$ is a group. Then every element
    $g$ has a unique inverse. That means if $g_1,g_2$
    are inverses of $g$, then $g_1=g_2$
\end{lemma}

\begin{proof}
    Here is the nice argument and as you can observe we only
    need to assume that $g_1\cdot g=e_G$ and $g\cdot g_2=e_G$.
    \[\begin{split}
        g_1
        &=g_1\cdot e_G\\
        &=g_1\cdot (g\cdot g_2)\\
        &=(g_1\cdot g)\cdot g_2\\
        &=e_G\cdot g_2\\
        &=g_2.
    \end{split}\]
    So the proof is finished.
\end{proof}

The inverse of $g\in G$ in a multiplicative notation
is denoted by $g^{-1}$. When we are working with an additive
notation $(G,+)$, the neutral element is denoted by $0$
and the inverse of $g\in G$ is denoted by $-g$.

\begin{lemma}
    Suppse $(G,\cdot)$ is a group. Then for every $g,h$ in $G$,
    we have $(g\cdot h)^{-1}=h^{-1}\cdot g^{-1}$.
\end{lemma}

\begin{proof}
    Since inverse of an element is unique, it is enough to check
    that $(g\cdot h)\cdot (h^{-1}\cdot g^{-1})=(h^{-1}\cdot g^{-1})\cdot (g\cdot h)=e_G$.
    \[\begin{split}
        (g\cdot h)\cdot (h^{-1}\cdot g^{-1})
        &=g\cdot (h\cdot h^{-1})\cdot g^{-1}\\
        &=(g\cdot e_G)\cdot g^{-1}\\
        &=g\cdot g^{-1}=e_G.
    \end{split}\]
    Similarly,
    \[\begin{split}
        (h^{-1}\cdot g^{-1})\cdot (g\cdot h)
        &=h^{-1}\cdot (g^{-1}\cdot g)\cdot h\\
        &=h^{-1}\cdot e_G\cdot h\\
        &=h^{-1}\cdot h=e_G.
    \end{split}\]
    Therefore $(g\cdot h)^{-1}=h^{-1}\cdot g^{-1}$.
\end{proof}

\begin{lemma}
    For every $g\in G$, $(g^{-1})^{-1}=g$.
\end{lemma}

\begin{proof}
    We have that $g^{-1}\cdot g=e_G$. Multiply both sides by $(g^{-1})^{-1}$
    from left to get 
    $\left((g^{-1})^{-1}\cdot g^{-1}\right)\cdot g=(g^{-1})^{-1}\cdot e_G=(g^{-1})^{-1}$.
    Hence $e_G\cdot g=(g^{-1})^{-1}$, and so $g=(g^{-1})^{-1}$.
\end{proof}

\begin{lemma}[Cancellation law]
    $g\cdot h=g\cdot h'\Rightarrow h=h'$. Similarly,
    $h\cdot g=h'\cdot g\Rightarrow h=h'$.
\end{lemma}

\begin{proof}
    $g\cdot h=g\cdot h'\Rightarrow g^{-1}\cdot (g\cdot h)=g^{-1}\cdot (g\cdot h')\Rightarrow h=h'$.
    The other is similar.
\end{proof}

Suppose $(G,\cdot)$ is a group and $g\in G$. For a positive integer $n$,
we let $g^n:=\underbrace{g\cdot\cdots\cdot g}_{n\text{ times}}$. For a negative 
integer $n$, we let $g^n:=\underbrace{(g^{-1})\cdot\cdots\cdot (g^{-1})}_{-n\text{ times}}$.
And we let $g^0=e_G$ (the neutral element).

\begin{lemma}
    For $n,m\in\FZ$, $(g^n)^m=g^{nm}$.
\end{lemma}

\begin{proof}
    We will consider various cases depending on signs of $m$ and $n$.
    Suppose $m$ and $n$ are positive. Then 
    \[(g^n)^m=\underbrace{g^n\cdots g^n}_{m\text{ times}}=(\overbrace{g\cdots g}^{n\text{ times}})\cdot\cdots\cdot(\overbrace{g\cdots g}^{n\text{ times}})=\overbrace{g\cdots g}^{mn\text{ times}}=g^{mn}.\]
    $m>0,n<0$
\end{proof}