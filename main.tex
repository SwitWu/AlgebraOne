\documentclass{article}
\usepackage{amsmath,amssymb}
\usepackage{amsthm}
\theoremstyle{plain}
\newtheorem{lemma}{Lemma}
\newtheorem{theorem}{Theorem}
\newtheorem{proposition}{Proposition}
\newtheorem{corollary}{Corollary}
\theoremstyle{definition}
\newtheorem*{definition}{Definition}
\newtheorem{example}{Example}

\newcommand{\R}{\mathrm{R}}
\newcommand{\FR}{\mathbb{R}}
\newcommand{\FC}{\mathbb{C}}
\newcommand{\FQ}{\mathbb{Q}}
\newcommand{\FZ}{\mathbb{Z}}
\newcommand{\modnequiv}{\overset{n}{\equiv}}
\renewcommand{\Im}{\operatorname{Im}}

\newenvironment{solve}%
               {\renewcommand{\proofname}{Solution}\begin{proof}}%
               {\end{proof}}

\begin{document}

\section{Introduction}


Historically algebra was created to understand zeros of
polynomial equations. Along the way the importance of various
system of numbers and their symmetries became evident.
The importance of symmetries of objects in other parts of
math and other sciences turned it into an important part of
algebra which has connections with geometry, analysis,
combinatorics, topology, etc. Symmetries of objects are studied
in group theory, which is the main subject of our course.

We start by recalling some of the basic concepts from
set theory and congruence arithmetic.

\paragraph{Equivalence Relation}
Let $X$ be a non-empty set. A relation over $X$ is a subset
$\R$ of $X\times X$. If $(x,y)\in\R$, we say $x$ is $\R$-related
to $y$ and write $x\R y$. Suppose $\R$ is a relation over $X$.
Then:
\begin{itemize}
    \item $\R$ is called reflexive if $\forall x\in X$, $x\R x$.
    \item $\R$ is called symmetric if $\forall x,y\in X$, $x\R y\Rightarrow y\R x$.
    \item $\R$ is called transitive if $\forall x,y\in X$, $x\R y,y\R z\Rightarrow x\R z$.
\end{itemize}

$\R$ is called an equivalent relation if $\R$ is reflexive, symmetric
and transitive.

Equivalent relations are essentially about equality with
respect to certain measurment. The following example
is an important indication of this concept:

\begin{example}
    Suppose $X$ and $Y$ are two non-empty sets and $f:X\to Y$
    is a function. Let $\sim$ be the following relation over $X$:
    \[\forall x_1,x_2\in X,x_1\sim x_2\Longleftrightarrow f(x_1)=f(x_2).\]
    Then $\sim$ is an equivalent relation.
\end{example}

Alternatively equivalent relations partition $X$ into subsets
that share the same property. For instance in the previous example
the shared property is having the same value under the function $f$.
Let's recall that $P$ is called a partition of a non-empty set
$X$, if 
\begin{itemize}
    \item $P$ consists of non-empty subsets of $X$, (subsets)
    \item $A,B\in P$ and $A\neq B\Rightarrow A\cap B=\varnothing$, (disjoint)
    \item $\forall x\in X$, $\exists A\in P$ such that $x\in A$. (covering)
\end{itemize}

Suppose $P$ is a partition of $X$. Then we get a classification function
from $X$ to $P$: $X\to P$, $x\mapsto [x]_P$ where $[x]_P$ is the 
unique element of $P$ which contains $x$. Notice that because of
the covering condition $x$ is contained in some element of $P$,
and because of the disjointness condition $x$ is in a unique
element of $P$. By the previous example, $x\sim y\Leftrightarrow [x]_P=[y]_P$
is an equivalent relation. So we can obtain the following lemma:

\begin{lemma}
    Suppose $P$ is a partition of a non-empty set $X$.
    For $x,y\in X$, $x\sim y$ if $x$ and $y$ are in the same element
    of $P$. Then $\sim$ is an equivalent relation.
\end{lemma}

\begin{proof}
    For $x\in X$, let $[x]_P$ be the unique element of $P$
    which contains $x$. So $x\mapsto[x]_P$ is a function from $X$
    to $P$. By the previous example, $x\sim y\Leftrightarrow[x]_P=[y]_P$
    is an equivalent relation over $X$. Notice that this means 
    $x\sim y$ exactly when $x$ and $y$ are in the same element of $P$.
\end{proof}

Starting with an equivalent relation $\sim$ over a non-empty
set $X$, we can partition $X$ with respect to $\sim$,
as we show next.

For $x\in X$, let $[x]:=\{y\in X\mid y\sim x\}$ (all the elements
that are $\sim$-related to $x$) We call $[x]$ the equivalent class
of $x$ with respect to $\sim$. When $x\sim y$, we say $x$ is equivalent to
$y$ with respect to $\sim$.

\begin{proposition}
    Suppose $\sim$ is an equivalent relation over a non-empty
    set $X$. Then $\{[x]\mid x\in X\}$ is a partition of $X$.
\end{proposition}

\begin{lemma}
    $x\sim y\Leftrightarrow [x]=[y]$.
\end{lemma}

\begin{proof}[Proof of lemma]
    ($\Leftarrow$) We want to show $[x]=[y]\Rightarrow x\sim y$.
    $x\sim x\Rightarrow x\in[x]\Rightarrow x\in[y]\Rightarrow x\sim y$.

    ($\Rightarrow$) $x\sim y\Rightarrow[x]=[y]$.
    To show equality of sets $[x]$ and $[y]$, it is necessary
    and sufficient to prove $[x]\subset [y]$ and $[y]\subset [x]$.
    Let's start by proving $[x]\subset [y]$.
    $\forall z\in[x]$, $z\sim x$. Combining with $x\sim y$
    we have $z\sim y$, so $z\in[y]$. Hence $[x]\subset [y]$.
    This means we showed
    \begin{equation}
        x\sim y\Rightarrow [x]\subset [y].\tag{I}
    \end{equation}
    Notice that $x\sim y\Rightarrow y\sim x$. Therefore,
    by (I), $[y]\subset [x]$.
    \begin{equation}
        x\sim y\Rightarrow y\sim x\Rightarrow [y]=[x].\tag{II}
    \end{equation}
    By (I) and (II), we have $x\sim y\Rightarrow [x]=[y]$.
\end{proof}


\begin{proof}[Proof of Proposition]
    $\forall x\in X$, $x\sim x$. Thus $x\in[x]$.
    This implies that $[x]$s are non-empty subsets and they cover
    $X$. Next we show the disjointness property.
    Suppose $z\in[x]\cap[y]$.
    \[\left.\begin{array}[]{c}
        z\in[x]\Rightarrow z\sim x\Rightarrow [z]=[x]\\
        z\in[y]\Rightarrow z\sim y\Rightarrow [z]=[y]
    \end{array}\right\}\Rightarrow [x]=[y].\]
    We showed that $[x]\cap[y]\neq\varnothing\Rightarrow [x]=[y]$.
    The contrapositive of this statement is 
    \[[x]\neq [y]\Rightarrow[x]\cap[y]=\varnothing,\]
    which is the disjointness property.
\end{proof}


Next we recall the congruence modulo $n$ relation which 
plays an important role in our course.
\section{Congruences}


The set of integers is denoted by $\FZ$. For $a,b\in\FZ$,
we say $a$ divide $b$ and write $a\mid b$ if $b=ak$ for some
integer $k$. Suppose $n$ is a non-zero integer, we say
$a$ is congruent to $b$ modulo $n$ and write $a\equiv b\pmod n$
or $a\modnequiv b$ if $n\mid a-b$.

\begin{lemma}
    $\modnequiv$ is an equivalent relation over $\FZ$.
\end{lemma}

As we have seen eariler, every equivalent relation
gives us a partition and an equality function.
For $a\in\FZ$, the equivalent class of $a$ with respect
to $\modnequiv$ is called the mod-$n$ residue class of $a$
and it is denoted by $[a]_n$. By the results that we 
proved for equivalent relations we have that
\begin{itemize}
    \item $\{[a]_n\mid a\in\FZ\}$ is a partiton of $\FZ$
    \item $a\modnequiv b\Leftrightarrow [a]_n=[b]_n$.
\end{itemize}

The partition $\{[a]_n\mid a\in\FZ\}$ is denoted by $\FZ_n$
and it is called the set of integers modulo $n$.
Notice that
\[\begin{split}
    b\in[a]_n
    &\Longleftrightarrow b\modnequiv a\\
    &\Longleftrightarrow n\mid b-a\\
    &\Longleftrightarrow \exists k\in\FZ,b-a=nk\\
    &\Longleftrightarrow \exists k\in\FZ,b=a+nk\\
    &\Longleftrightarrow b\in\{a+nk\mid k\in\FZ\}
\end{split}\]

To understand the set $\FZ_n$ better,
we recall the well-ordering principle and
the long division. One of the important properties
of positive integers is the well-ordering principle.

\paragraph{The well-ordering principle}
Every non-empty subset of the set $\FZ^{\geq 0}$
of non-negative integers has a minimum.

Using the well-ordering principle we can prove the
division algorithm.

\begin{theorem}[The division algorithm]
    For every $a\in\FZ$, $b\in\FZ\backslash\{0\}$,
    there is a unique pair $(q,r)$ of integers
    such that $a=bq+r$ and $0\leq r<|b|$.
\end{theorem}

Suppose the pair $(q,r)$ is given in the long division
algorithm. Then $q$ is called the quotient of $a$ divided by 
$b$ and $r$ is called the remainder of $a$ divided by $b$.

Using the long division algorithm we obtain that $\FZ_n$
has $n$ elements.

\begin{proposition}
    Suppose $n$ is an integer more than $1$. Then 
    \[\FZ_n=\{[0]_n,[1]_n,\dots,[n-1]_n\}\text{\quad and\quad}|\FZ_n|=n.\]
\end{proposition}

\begin{proof}
    For every $a\in\FZ$, by the long division algorithm,
    there are integers $q,r$ such that $a=nq+r$ and $0\leq r<n$.
    $a=nq+r$ implies that $a\modnequiv r$. Hence $[a]_n=[r]_n$.
    Thus $[a]_n\in\{[0]_n,[1]_n,\dots,[n-1]_n\}$. Next we show $[i]_n\neq [j]_n$
    if $0\leq j<i\leq n-1$. Suppose to the contrary $[i]_n=[j]_n$.
    \[\begin{split}
        [i]_n=[j]_n&\Rightarrow i\modnequiv j\Rightarrow n\mid i-j \\
                   &\Rightarrow i-j=nq\text{\quad for some integer }q
    \end{split}\]
    Notice that $0\leq j<i<n$ implies $0<i-j<n$, therefore $0<nq<n$.
    Then $0<q<1$ which is a contradiction as there is no integer more than
    $0$ and less than $1$. This completes the proof.
\end{proof}

The set of integers modulo $n$ have arithmetic operations similar to the set
of integers. This can be viewed as a generalization of the fact that
\begin{table}
    \centering
    \begin{tabular}{c|cc}
        $+$ & even & odd \\
        \hline
        even & even & odd \\
        odd & odd & even
    \end{tabular}
    \quad
    \begin{tabular}{c|cc}
        $\cdot$ & even & odd \\
        \hline
        even & even & even \\
        odd & even & odd
    \end{tabular}
\end{table}

No matter what even or odd numbers we pick the above
tables hold. So we can simply view the above tables as
operations for the set of even numbers and the set of odd numbers.

\begin{lemma}
    The following are well-defined operations on $\FZ_n$:
    \[[a]_n+[b]_n=[a+b]_n\text{\quad and\quad}[a]_n\cdot [b]_n=[a\cdot b]_n.\]
\end{lemma}

Before we go to the proof of this lemma, let's try to understand 
what it says. Notice that $[a]_n$ is a set and $a\in[a]_n$.
We say $a$ is a representative of this residue class.
Recall that $[a]_n$ is the equivalence class of $a$
with respect to the equivalent relation $\modnequiv$.
From all the elements of $\FZ$ that are in the same class
we are choosing a representative. We are doing the same for
the residue class $[b]_n$. Then we are adding the chosen representatives
in $\FZ$, and next we are considering the residue class of the
sum (or product) of these representatives.
A priori it is not clear why these operations do not depend
on the choice of the representatives. This is an extremely
important process. We will be using the same idea later when we define
factor groups. Whenever you are using a representative from a 
class and applying certain logic or operations to obtain
a claim for the entire class, you have to be extra careful.
You have to make sure that you are not ``stareotyping'' and
your process is independent of the choice of a representative.

\begin{proof}
    Suppose $[a]_n=[a']_n$ and $[b]_n=[b']_n$. We have to show that
    \[[a+b]_n=[a'+b'_n]\text{\quad and\quad}[a\cdot b]_n=[a'\cdot b']_n.\]
    \begin{equation}
        [a]_n=[a']_n\Rightarrow a\modnequiv a'\Rightarrow\exists k\in\FZ,a-a'=nk\tag{I}
    \end{equation}
    \begin{equation}
        [b]_n=[b']_n\Rightarrow b\modnequiv b'\Rightarrow\exists l\in\FZ,b-b'=nl\tag{II}
    \end{equation}
    Therefore,
    \[\begin{split}
        (I)+(II)
        &\Rightarrow (a-a')+(b-b')=nk+nl=n(k+l)\\
        &\Rightarrow (a+b)-(a'+b')=n(k+l)\\
        &\Rightarrow a+b\modnequiv a'+b'\\
        &\Rightarrow [a+b]_n=[a'+b']_n.
    \end{split}\]

    Next we want to show $[a\cdot b]_n=[a'\cdot b']_n$. Notice
    that 
    \[\begin{split}
        [a\cdot b]_n=[a'\cdot b']_n
        &\Leftrightarrow a\cdot b\modnequiv a'\cdot b'\\
        &\Leftrightarrow a\cdot b-a'\cdot b'\text{ is a multiple of }n
    \end{split}\]
    We change one factor at a time. This is similar to how we show
    the product rule in caculus.
    \[\begin{split}
        a\cdot b-a'\cdot b'
        &=a\cdot b-a'\cdot b+a'\cdot b-a'\cdot b'\\
        &=(a-a')\cdot b+a'\cdot (b-b')\\
        &=(nk)\cdot b+a'\cdot(nl)\\
        &=n\cdot (kb+a'l).
    \end{split}\]
    Therefore $a\cdot b\modnequiv a'\cdot b'$, so $[a\cdot b]_n=[a'\cdot b']_n$.
\end{proof}

Next we see that the above operations on $\FZ_n$
satisfy associativity and distribution. Moreover
$\FZ_n$ has neutral elements with respect to both $+$ and $\cdot$.
Every element has an additive inverse. Later we will see
what elements have multiplicative inverse.

\begin{proposition}
    For every $[a]_n$, $[b]_n$, $[c]_n\in\FZ_n$
    we have 
    \begin{description}
        \item[Associative] \[[a]_n+([b]_n+[c]_n)=([a]_n+[b]_n)+[c]_n\] \[[a]_n\cdot([b]_n\cdot[c]_n)=([a]_n\cdot [b]_n)\cdot[a]_n\]
        \item[Neutral element] \[[a]_n+[0]_n=[0]_n+[a]_n=[a]_n\] \[[a]_n\cdot [1]_n=[1]_n\cdot [a]_n=[a]_n\]
        \item[Additive inverse]  \[[a]_n+[-a]_n=[-a]_n+[a]_n=[0]_n\]
        \item[Commutative] \[[a]_n+[b]_n=[b]_n+[a]_n\] \[[a]_n\cdot [b]_n=[b]_n\cdot [a]_n\]
        \item[Distributive] \[[a]_n\cdot([b]_n+[c]_n)=[a]_n\cdot [b]_n+[a]_n\cdot [c]_n\] \[([b]_n+[c]_n)\cdot [a]_n=[b]_n\cdot [a]_n+[c]_n\cdot [a]_n\] 
    \end{description} 
\end{proposition}
All the claims are straightforward conclusions of similar
properties for integers.
\section{Greatest common divisor}

To understand what elements of $\FZ_n$ have multiplicative
inverse, we need to recall basic properties of greatest common
divisor of integers. In particular, we recall Euclid's
algorithm.

The greatest common divisor of two non-zero integers $a$
and $b$ is, as its name suggests,
\[\max\{d\in\FZ\mid d\mid a,d\mid b\},\]
and it is denoted by $\gcd(a,b)$.

Notice that if $a$ is a non-zero integer and $d\mid a$,
then $d\leq |a|$. Hence $\gcd(a,b)\leq\min\{|a|,|b|\}$
if $a$, $b$ are two non-zero integers.

\begin{lemma}
    Suppose $a,b,d\in\FZ$. Then
    \begin{itemize}
        \item $d\mid a$, $d\mid b\Longrightarrow d\mid ra+sb$ for every $r,s\in\FZ$.
        \item $d\mid b$, $d\mid a-b\Longrightarrow d\mid a$.
    \end{itemize}
\end{lemma}

\begin{corollary}
    Suppose $a,b$ are two positive integers. Then
    \[\gcd(a,b)=\gcd(b,a-b).\]
\end{corollary}

\begin{proof}
    We show that $d$ is a common divisor of $a$ and $b$
    if and only if $d$ is a common divisor of $b$ and $a-b$.
\end{proof}

Next we point out the connection with Euclid's algorithm.

\begin{lemma}
    Suppose $n$ is a non-zero integer. If $a\modnequiv a'$, then
    \[\gcd(a,n)=\gcd(a',n).\]
\end{lemma}

\begin{proof}
    Since $a\modnequiv a'$, $a-a'=nk$ for some $k\in\FZ$.
    \begin{equation}
        d\mid a\text{ and }d\mid a'\Rightarrow d\mid kn+1a'\Rightarrow d\mid a\tag{I}
    \end{equation}
    \begin{equation}
        d\mid n\text{ and }d\mid a\Rightarrow d\mid 1a+(-k)n\Rightarrow d\mid a'\tag{II}
    \end{equation}
    By (I), (II), $\{d\in\FZ\mid d\mid n,d\mid a\}=\{d\in\FZ\mid d\mid n,d\mid a'\}$.
    Hence $\gcd(n,a)=\gcd(n,a')$.
\end{proof}
Euclid's algorithm is a fast way of finding the gcd of two
positive integers. Similar to the pictorial method,
Euclid's algorithm gives us a process through which the gcd 
stays the same, but we get smaller and smaller pairs.

Suppose $a\geq b$ are two positive integers. Let
$a_0:=a$, $a_1:=b$. We divide $a_0$ by $a_1$, that is
$a_0=a_1q_1+a_2$. Then $a_0\overset{a_1}{\equiv}a_2$,
ans so by the above lemma, $\gcd(a_0,a_1)=\gcd(a_1,a_2)$.
Next we divide $a_1$ by $a_2$ if $a_2\neq 0$,
and repeat this process till the remainder is $0$.
\[\begin{matrix}
    a_0=a_1q_1+a_2, & \gcd(a_0,a_1)=\gcd(a_1,a_2), & a_1>a_2 \\
    a_1=a_2q_2+a_3, & \gcd(a_1,a_2)=\gcd(a_2,a_3), & a_2>a_3 \\
    \vdots          & \vdots                       & \vdots \\
    a_{n-1}=a_nq_n+0,&\gcd(a_{n-1},a_n)=a_n,       & a_n>0
\end{matrix}\]
Hence $a_0\geq a_1>a_2>\cdots>a_n>0$ and $a_n=\gcd(a_0,a_1)=\gcd(a,b)$.
Notice that, for every $0\leq i<n$,
\[\begin{pmatrix}
    0 & 1 \\ 1 & -q_i
\end{pmatrix}\begin{pmatrix}
    a_{i-1} \\ a_i
\end{pmatrix}=\begin{pmatrix}
    a_i \\ a_{i-1}-a_iq_i
\end{pmatrix}=\begin{pmatrix}
    a_i \\ a_{i+1}
\end{pmatrix}.\]
Hence
\[\begin{pmatrix}
    a_n \\ 0
\end{pmatrix}=\begin{pmatrix}
    0 & 1 \\ 1 & -q_n
\end{pmatrix}\begin{pmatrix}
    0 & 1 \\ 1 & -q_{n-1}
\end{pmatrix}\cdots\begin{pmatrix}
    0 & 1 \\ 1 & -q_1
\end{pmatrix}\begin{pmatrix}
    a_0 \\ a_1
\end{pmatrix}\]
Therefore $a_n=ra_0+sa_1$ for some integers $r,s$.

\begin{theorem}
    For every non-zero integers $a$ and $b$, there are integers
    $r$ and $s$ such that $\gcd(a,b)=ra+sb$.
\end{theorem}

\begin{proof}
    We notice that $\gcd(a,b)=\gcd(|a|,|b|)$. Now claim
    follows from the above process.
\end{proof}

Here we review basic properties of gcd of two integers.

\begin{theorem}
    Suppose $a,b$ are two non-zero integers. The if $\gcd(a,b)=d$,
    then $\gcd\left(\frac{a}{d},\frac{b}{d}\right)=1$.
\end{theorem}

\begin{proof}
    Since $\gcd(a,b)=d$, $d\mid a$ and $d\mid b$ and
    \[d=ra+sb\text{ for some }r,s\in\FZ.\]
    Hence $\frac{a}{d},\frac{b}{d}\in\FZ$ and $1=r\left(\frac{a}{d}\right)+s\left(\frac{b}{d}\right)$.

    Let $d':=\gcd\left(\frac{a}{d},\frac{b}{d}\right)$. Then $d'\mid\frac{a}{d}$, $d'\mid\frac{b}{d}$,
    and so $d'\mid r\left(\frac{a}{d}\right)+s\left(\frac{b}{d}\right)$. Therefore
    $d'\mid 1$, which means $\gcd\left(\frac{a}{d},\frac{b}{d}\right)=1$.
\end{proof}

\begin{theorem}
    Suppose $a,b$ are two non-zero integers. If $d:=\gcd(a,b)$
    and $d'$ is a common divisor of $a$ and $b$, then $d'\mid d$.
\end{theorem}

\begin{proof}
    Since $d=\gcd(a,b)$, $d=ra+sb$ for some $r,s\in\FZ$.
    Because $d'\mid a$ and $d'\mid b$, $d'\mid ra+sb$. Hence $d'\mid d$.
\end{proof}

\begin{theorem}
    Suppose $a,b,c$ are three non-zero integers.
    Then $\gcd(ac,bc)=|c|\gcd(a,b)$.
\end{theorem}

\begin{proof}
    Suppose $d:=\gcd(a,b)$. Then $d\mid a$ and $d\mid b$,
    and so $d|c|$ divides $ac$ and $d|c|$ divides $bc$. Hence
    \[d|c|\leq\gcd(ac,bc).\]
    On the other hand, $d=\gcd(a,b)$ implies that $d=ra+sb$
    for some $r,s\in\FZ$. Hence
    \[d|c|=ra|c|+sb|c|=\pm(r(ac)+s(bc)).\]
    Notice that $\gcd(ac,bc)$ divides every integer linear combination
    of $ac$ and $bc$. Hence
    \[\gcd(ac,bc)\mid d|c|.\]
    Therefore $\gcd(ac,bc)=d|c|$, which means
    \[\gcd(ac,bc)=|c|\gcd(a,b).\qedhere\]
\end{proof}

\begin{theorem}[Euclid's lemma]
    For $a,b,c\in\FZ\backslash\{0\}$, if $\gcd(a,b)=1$ and $a\mid bc$,
    then $a\mid c$.
\end{theorem}

\begin{proof}
    Since $\gcd(a,b)=1$,
    \[1=ra+sb\text{ for some }r,s\in\FZ.\]
    Multiplying both sides of the above equality by $c$,
    we obtain that
    \[c=rc(a)+s(bc).\]
    Since $a\mid a$ and $a\mid bc$, $a$ divides every integer linear
    combination of $a$ and $bc$. Therefore $a\mid c$.
\end{proof}
\section{Multiplicative structure of integers mod $n$}
Here we want to investigate what elements of $\mathbb{Z}_n$
have multiplicative inverse.
\begin{definition}
    We say $[a]_n\in\mathbb{Z}_n$ has a multiplicative inverse
    if $[a]_n\cdot [a']_n=[1]_n$ for some $[a']_n\in\mathbb{Z}_n$. We say $[a]_n$
    is a unit of $\mathbb{Z}_n$ if it has a multiplicative 
    inverse. The set of all the units of $\mathbb{Z}_n$ is denoted
    by $\mathbb{Z}_n^*$.
\end{definition}

\begin{theorem}
    Suppose $n\in\mathbb{Z}$ and $n\geq 2$. Then
    \[\mathbb{Z}_n^*=\{[a]_n\mid \gcd(a,n)=1\}.\]
    Moreover $|\mathbb{Z}_n^*|=|\{a\in\mathbb{Z}\mid 1\leq a\leq n,\gcd(a,n)=1\}|$.
\end{theorem}

(The left hand side of the above equality is denoted
by $\phi(n)$ and it is called \textbf{Euler's phi function}.)

\begin{proof}
    ($\subset$) Suppose $[a]_n\in\mathbb{Z}_n^*$. Then $[a]_n[a']_n=[1]_n$
    for some $a'\in\mathbb{Z}$. Hence $[aa']_n=[1]_n$ which implies
    that $aa'\equiv 1\pmod n$. (Earlier we proved that $b\equiv b'\pmod n$
    implies $\gcd(b,n)=\gcd(b',n)$). Hence $\gcd(aa',n)=\gcd(1,n)=1$.
    Therefore $\gcd(a,n)=1$.

    ($\supset$) Suppose $\gcd(a,n)=1$. Then $1=ra+sn$
    for some $r,s\in\mathbb{Z}$. Since $ra+sn=1$, we obtain that
    \[ra\equiv 1\pmod n.\]
    This implies that $[ra]_n=[1]_n$, and so 
    $[r]_n[a]_n=[1]_n$. Therefore $[a]_n\in\mathbb{Z}_n$.
\end{proof}

\begin{example}
    List all the elements of $\FZ_6^*$
\end{example}

\begin{solve}
    $\FZ_6^*=\{[a]_6\mid 1\leq a\leq 6,\gcd(a,6)=1\}=\{[1]_6,[5]_6\}$.
\end{solve}

\begin{example}
    List all the elements of $\FZ_8^*$.
\end{example}

\begin{proof}
    $\FZ_8^*=\{[1]_8,[3]_8,[5]_8,[7]_8\}$.
\end{proof}

\begin{proposition}
    Suppose $p$ is prime. Then $\FZ_p^*=\FZ_p\backslash\{[0]_p\}$.
\end{proposition}

\begin{proof}
    By the previous theorem,
    \[\FZ_p^*=\{[a]_p\mid 1\leq a\leq p,\gcd(a,p)=1\}.\]
    Since $p$ is prime, for every integer $1\leq a<p$ we have
    $\gcd(a,p)=1$. Hence $\FZ_p^*=\{[a]_p\mid 1\leq a<p\}$.
    Since $\FZ_p=\{[0]_p,[1]_p,\dots,[p-1]_p\}$,
    we obtain that $\FZ_p^*=\FZ_p\backslash\{[0]_p\}$.
\end{proof}

The converse of the previous proposition is essentially true:

\begin{proposition}
    Suppose $n\in\FZ$, $n\geq 2$. If $\FZ_n^*=\FZ_n\backslash\{[0]_n\}$,
    then $n$ is prime.
\end{proposition}

\begin{proof}
    If $\FZ_n^*=\FZ_n\backslash\{[0]_n\}$, then $\phi(n)=n-1$.
    This means
    \[|\{a\in\FZ\mid 1\leq a\leq n,\gcd(a,n)=1\}|=n-1.\]
    So $n$ does not have any divisor in the interval $(1,\dots,n)$.
    Since $n\geq 2$, we deduce that $n$ is prime.
\end{proof}

\begin{example}
    Suppose $p$ is prime and $k\in\FZ$. Then $\phi(p^k)=p^k-p^{k-1}$.
\end{example}

\begin{solve}
    We show that $\gcd(a,p^k)=1\Leftrightarrow p\nmid a$.

    ($\Rightarrow$) We show the contrapositive. If $p\mid a$,
    then $p$ is a common divisor of $a$ and $p^k$, and so $\gcd(a,p^k)\neq 1$.

    ($\Leftarrow$) We proceed by induction on $k$.

    Base case, $k=1$.

    Since $p\nmid a$, $\gcd(a,p)\neq p$. Since $p$ has exactly two positive
    divisors $1$ and $p$, we deduce that $\gcd(a,p)=1$.

    Induction step. $\gcd(a,p^k)=1\Rightarrow\gcd(a,p^{k+1})=1$.

    By the base case, $\gcd(a,p)=1$. Then $\gcd(d,p)=1$ where
    $d=\gcd(a,p^{k+1})$. Since $d\mid p^{k+1}$ and $\gcd(d,p)=1$,
    by Euclid's lemma, $d\mid p^k$. So $d$ is a common divisor of $a$
    and $p^k$. Hence $d\leq\gcd(a,p^k)$. By the induction hypothesis
    $\gcd(a,p^k)=1$, and so $d=1$ (Notice that $d\geq 1$). This means
    $\gcd(a,p^{k+1})=1$, and claim follows.

    By the above claim,
    \[\begin{split}
        \phi(p^k)
        &=|\{a\in\FZ\mid 1\leq a\leq p^k,\gcd(a,p^k)=1\}|\\
        &=|\{a\in\FZ\mid 1\leq a\leq p^k,p\nmid a\}|\\
        &=|[1\dots p^k]\backslash\{a\in\FZ\mid 1\leq a\leq p^k,p\mid a\}|\\
        &=p^k-|\{a\in\FZ\mid 1\leq a\leq p^k,p\mid a\}|.
    \end{split}\]
    \[\begin{split}
        1\leq a\leq p^k,p\mid a
        &\Longleftrightarrow a=pa'\text{ and }1\leq pa'\leq p^k\\
        &\Longleftrightarrow a=pa'\text{ and }1\leq a'\leq p^{k-1}
    \end{split}\]
    So there are $p^{k-1}$ many $a'$s that satisfy $1\leq a\leq p^k$ and $p\mid a$.
    Hence \[\phi(p^k)=p^k-p^{k-1}.\qedhere\]
\end{solve}

Next we show that $\FZ_n^*$ is closed under multiplication.
This type of property plays an important role in group theory.

\begin{theorem}
    Suppose $n\in\FZ$ and $n\geq 2$. Then 

    (Operator) For every $[a]_n,[b]_n\in\FZ_n^*$, $[a]_n\cdot [b]_n\in\FZ_n^*$.

    (Associative) For every $[a]_n,[b]_n,[c]_n\in\FZ_n^*$,
    \[([a]_n\cdot [b]_n)\cdot [c]_n=[a]_n\cdot([b]_n\cdot [c]_n).\]

    (Neutral element) For every $[a]_n\in\FZ_n^*$, $[a]_n\cdot[1]_n=[1]_n\cdot[a]_n=[a]_n$.

    (Inverse) For every $[a]_n\in\FZ_n^*$, there is $[a']_n\in\FZ_n^*$ such that
    \[[a]_n\cdot [a']_n=[a']_n\cdot [a]_n=[1]_n.\]
\end{theorem}

\begin{proof}
    We have already proved that multiplication in $\FZ_n$
    is associative and $[1]_n$ is a neutral element of multiplication.
    Next we show that $\FZ_n^*$ is closed under multiplication.
    Suppose $[a]_n,[b]_n\in\FZ_n^*$. Then there are 
    $[a']_n,[b']_n\in\FZ_n$ such that $[a]_n\cdot [a']_n=[1]_n$
    and $[b]_n\cdot [b']_n=[1]_n$. Hence
    \[\begin{split}
        ([a]_n\cdot [b]_n)([b']_n\cdot [a']_n)
        &=[a]_n([b]_n\cdot [b']_n)[a']_n\\
        &=([a]_n\cdot [1]_n)[a']_n\\
        &=[a]_n\cdot [a']_n=[1]_n.
    \end{split}\]
    This means $[a]_n\cdot [b]_n\in\FZ_n^*$. Finally let's us discuss
    why every element of $\FZ_n^*$ has an inverse in $\FZ_n^*$.

    Since $[a]_n\in\FZ_n^*$, there is $[a']_n$ in $\FZ_n$
    such that 
    \[[a]_n\cdot [a']_n=[1]_n.\]
    The above equality implies that $[a']_n\cdot [a]_n=[1]_n$,
    and so $[a']_n\in\FZ_n^*$. This completes the proof.
\end{proof}
\section{Group}
Group theory is (mostly) about symmetrices of objects.
In some interesting examples in geometry, combinatorics,
or even chemistry, knowing the symmetrices uniquely 
determine the object. One can say that at a meta-level,
the whole mathematics (and in general sciences) is about
finding patterns as we want to reduce the amount of
data that we need to store. (Lowering the complexity
of the objects that we are studying.)

We start with an axiomatic definition of groups, and then 
give the relation with symmetrices.
\begin{definition}
    Suppose $G$ is a non-empty set and $(g_1,g_2)\mapsto g_1\cdot g_2$
    is an operator on $G$ (that means it is a function from $G\times G$
    to $G$). We say $(G,\cdot)$ (or simply $G$) is a group if the 
    following properties hold.
    \begin{itemize}
        \item (Associative) $\forall g_1,g_2,g_3\in G$, $g_1\cdot (g_2\cdot g_3)=(g_1\cdot g_2)\cdot g_3$
        \item (Neutral element) $\exists e\in G$, $\forall g\in G$, $g\cdot e=e\cdot g=g$
        \item (Inverse) $\forall g\in G$, $\exists g'\in G$, $g\cdot g'=g'\cdot g=e$, where $e$ is a neutral element.
    \end{itemize}
\end{definition}

We have already seen some exmaples of groups.
\begin{example}
    $(\mathbb{Z},+)$, $(\mathbb{Q},+)$, $(\FR,+)$ and $(\mathbb{C},+)$ are groups.
\end{example}

\begin{example}
    $(\mathbb{Q}\backslash\{0\},\cdot)$, $(\FR\backslash\{0\},\cdot)$, and 
    $(\mathbb{C}\backslash\{0\},\cdot)$ are groups.
\end{example}

\begin{example}
    For every integer $n\geq 2$, $(\mathbb{Z}_n,+)$ is a group.
\end{example}

\begin{example}
    For every integer $n\geq 2$, $(\mathbb{Z}_n^*,\cdot)$ is a group.
\end{example}

In some of the examples, we showed the uniqueness
of a neutral element when it exists. Next we show
this property in a general setting.

\begin{lemma}
    Suppose $G$ is a non-empty set, and $(g_1,g_2)\mapsto g_1\cdot g_2$
    is an operation. Suppose $e,e'\in G$ are neutral
    elements of $\cdot$. Then $e=e'$. In particular,
    in a group, there is a unique neutral element.
\end{lemma}

\begin{proof}
    Since $e$ is a neutral element, $e\cdot e'=e'$.
    Because $e'$ is a neutral element, $e\cdot e'=e$.
    Altogether we have 
    \[e'=e\cdot e'=e'.\qedhere\]
\end{proof}

Next we show the uniqueness of inverse in a group.

\begin{lemma}
    Suppose $(G,\cdot)$ is a group. Then every element
    $g$ has a unique inverse. That means if $g_1,g_2$
    are inverses of $g$, then $g_1=g_2$
\end{lemma}

\begin{proof}
    Here is the nice argument and as you can observe we only
    need to assume that $g_1\cdot g=e_G$ and $g\cdot g_2=e_G$.
    \[\begin{split}
        g_1
        &=g_1\cdot e_G\\
        &=g_1\cdot (g\cdot g_2)\\
        &=(g_1\cdot g)\cdot g_2\\
        &=e_G\cdot g_2\\
        &=g_2.
    \end{split}\]
    So the proof is finished.
\end{proof}

The inverse of $g\in G$ in a multiplicative notation
is denoted by $g^{-1}$. When we are working with an additive
notation $(G,+)$, the neutral element is denoted by $0$
and the inverse of $g\in G$ is denoted by $-g$.

\begin{lemma}
    Suppse $(G,\cdot)$ is a group. Then for every $g,h$ in $G$,
    we have $(g\cdot h)^{-1}=h^{-1}\cdot g^{-1}$.
\end{lemma}

\begin{proof}
    Since inverse of an element is unique, it is enough to check
    that $(g\cdot h)\cdot (h^{-1}\cdot g^{-1})=(h^{-1}\cdot g^{-1})\cdot (g\cdot h)=e_G$.
    \[\begin{split}
        (g\cdot h)\cdot (h^{-1}\cdot g^{-1})
        &=g\cdot (h\cdot h^{-1})\cdot g^{-1}\\
        &=(g\cdot e_G)\cdot g^{-1}\\
        &=g\cdot g^{-1}=e_G.
    \end{split}\]
    Similarly,
    \[\begin{split}
        (h^{-1}\cdot g^{-1})\cdot (g\cdot h)
        &=h^{-1}\cdot (g^{-1}\cdot g)\cdot h\\
        &=h^{-1}\cdot e_G\cdot h\\
        &=h^{-1}\cdot h=e_G.
    \end{split}\]
    Therefore $(g\cdot h)^{-1}=h^{-1}\cdot g^{-1}$.
\end{proof}

\begin{lemma}
    For every $g\in G$, $(g^{-1})^{-1}=g$.
\end{lemma}

\begin{proof}
    We have that $g^{-1}\cdot g=e_G$. Multiply both sides by $(g^{-1})^{-1}$
    from left to get 
    $\left((g^{-1})^{-1}\cdot g^{-1}\right)\cdot g=(g^{-1})^{-1}\cdot e_G=(g^{-1})^{-1}$.
    Hence $e_G\cdot g=(g^{-1})^{-1}$, and so $g=(g^{-1})^{-1}$.
\end{proof}

\begin{lemma}[Cancellation law]
    $g\cdot h=g\cdot h'\Rightarrow h=h'$. Similarly,
    $h\cdot g=h'\cdot g\Rightarrow h=h'$.
\end{lemma}

\begin{proof}
    $g\cdot h=g\cdot h'\Rightarrow g^{-1}\cdot (g\cdot h)=g^{-1}\cdot (g\cdot h')\Rightarrow h=h'$.
    The other is similar.
\end{proof}

Suppose $(G,\cdot)$ is a group and $g\in G$. For a positive integer $n$,
we let $g^n:=\underbrace{g\cdot\cdots\cdot g}_{n\text{ times}}$. For a negative 
integer $n$, we let $g^n:=\underbrace{(g^{-1})\cdot\cdots\cdot (g^{-1})}_{-n\text{ times}}$.
And we let $g^0=e_G$ (the neutral element).

\begin{lemma}
    For $n,m\in\FZ$, $(g^n)^m=g^{nm}$.
\end{lemma}

\begin{proof}
    We will consider various cases depending on signs of $m$ and $n$.
    Suppose $m$ and $n$ are positive. Then 
    \[(g^n)^m=\underbrace{g^n\cdots g^n}_{m\text{ times}}=(\overbrace{g\cdots g}^{n\text{ times}})\cdot\cdots\cdot(\overbrace{g\cdots g}^{n\text{ times}})=\overbrace{g\cdots g}^{mn\text{ times}}=g^{mn}.\]
    $m>0,n<0$
\end{proof}


\section{Homomorphism and subgroups}

Whenever we learn about a new structure in mathematics, we 
should study the functions between these objects that preserve
their properties. These functions are often called homomorphism.
(In a very vague sense homomorphisms give us a global understanding
of the objects.) Another point of view is from inside:
we often study subsets that share the same property. For
instance in linear algebra, the objects of interest are vector
spaces, the homomorphisms are linear maps, and subsets that
share the same properties are subspaces. We do the same for groups.

\begin{definition}
    Suppose $(G,\cdot)$ and $(H,*)$ are two groups.
    Then a function $f:G\to H$ is called a group homomorphism if 
    for every $g_1,g_2\in G$, $f(g_1\cdot g_2)=f(g_1)*f(g_2)$.
\end{definition}

\begin{definition}
    Suppose $(G,\cdot)$ is a group. Then a subset $K$ of $G$
    is called a subgroup of $G$ if $K$ is a group with respect to the
    operation $\cdot$.
\end{definition}

Next we see a few examples.

\begin{example}
    Suppose $n$ is an integer and $n\geq 2$. Then
    \[c_n:\mathbb{Z}\to\mathbb{Z}_n,c_n(a)=[a]_n\]
    is a group homomorphism.
\end{example}

\begin{proof}
    For $\forall a,b\in\mathbb{Z}$,
    \[c_n(a+b)=[a+b]_n=[a]_n+[b]_n=c_n(a)+c_n(b).\qedhere\]
\end{proof}

\begin{example}
    $f:\mathbb{Z}\to\mathbb{Z}$, $f(x)=-x$ is a group homomorphism.
\end{example}

\begin{proof}
    For every $x,y\in\mathbb{Z}$,
    \[f(x+y)=-(x+y)=(-x)+(-y)=f(x)+f(y).\qedhere\]
\end{proof}

\begin{example}
    Let $\FR^{>0}$ be the set of positive
    real numbers. Notice that $\FR^{>0}$ is a group
    under multiplication. Then 
    \[\ln:\FR^{>0}\to\FR\text{ is a group homomorphism.}\]
\end{example}

\begin{proof}
    For every $x,y\in\FR^{>0}$,
    \[\ln(x\cdot y)=\ln(x)+\ln(y).\qedhere\]
\end{proof}

\begin{example}
    Let $N:\FC\backslash\{0\}\to\FR^{>0}$, $N(z)=|z|$. Then 
    $N$ is a group homomorphism.
\end{example}

\begin{proof}
    For every $z\in\FC\backslash\{0\}$, $|z|\in\FR^{>0}$
    and $|z_1\cdot z_2|=|z_1|\cdot|z_2|$.
\end{proof}

\begin{example}
    Let $GL_n(\FR)$ be the set of invertible $n\times n$
    real matrices. From linear algebra we know that matrix
    multiplication is associative, product of two
    invertible $n\times n$ matrices is invertible, for
    every $a$ in $GL_n(\FR)$, $a\cdot I_n=I_n\cdot a=a$
    where $I_n$ is the identity matrix.
    So $(GL_n(\FR),\cdot)$ is a group. Let $\theta:GL_n(\FR)\to GL_n(\FR)$,
    $\theta(x)=(x^t)^{-1}$ where $x^t$ is the transpose of $x$. Then
    $\theta$ is a group homomorphism.
\end{example}

\begin{proof}
    \[\theta(x\cdot y)=((x\cdot y)^t)^{-1}=(y^t\cdot x^t)^{-1}=(x^t)^{-1}\cdot(y^t)^{-1}=\theta(x)\cdot\theta(y).\qedhere\]
\end{proof}

\begin{example}
    Suppose $(G,\cdot)$ is a group. Then $f:G\to G$, $f(g)=g^{-1}$
    is a group homomorphism if and only if $G$ is abelian.
\end{example}

\begin{proof}
    ($\Rightarrow$) For every $g,h\in G$, $f(g\cdot h)=f(g)\cdot f(h)$.
    Then 
    \begin{equation}
        (g\cdot h)^{-1}=g^{-1}\cdot h^{-1}\Rightarrow h^{-1}\cdot g^{-1}=g^{-1}\cdot h^{-1}\tag{I}
    \end{equation}
    For $x,y\in G$, let $g=x^{-1}$ and $h=y^{-1}$ in (I). Then
    we obtain $(y^{-1})^{-1}\cdot (x^{-1})^{-1}=(x^{-1})^{-1}\cdot(y^{-1})^{-1}$.\
    Since $(x^{-1})^{-1}=x$ and $(y^{-1})^{-1}=y$, we conclude
    $y\cdot x=x\cdot y$. Therefore $G$ is abelian.

    ($\Leftarrow$) $f(g\cdot h)=(g\cdot h)^{-1}=h^{-1}\cdot g^{-1}=f(h)\cdot f(g)=f(g)\cdot f(h)$.
\end{proof}

\begin{example}
    Suppose $(G,\cdot)$ is a group and $g\in G$. Let
    \[c_g:G\to G,c_g(x):=g\cdot x\cdot g^{-1}.\]
    Then $c_g$ is a group homomorphism.
\end{example}

\begin{proof}
    For $x,y\in G$, we have to show that 
    \[c_g(x\cdot y)=c_g(x)\cdot c_g(y).\]
    We have $c_g(x\cdot y)=g\cdot x\cdot y\cdot g^{-1}$ and 
    \[\begin{split}
        c_g(x)\cdot c_g(y)
        &=(g\cdot x\cdot g^{-1})\cdot(g\cdot y\cdot g^{-1})\\
        &=g\cdot x\cdot (g^{-1}\cdot g)\cdot y\cdot g^{-1}\\
        &=g\cdot x\cdot e_G\cdot y\cdot g^{-1}\\
        &=g\cdot x\cdot y\cdot g^{-1}
    \end{split}\]
    Therefore $c_g(x\cdot y)=c_g(x)\cdot c_g(y)$.
\end{proof}

\begin{example}
    Suppose $(G,\cdot)$ is a group and $g\in G$. Then
    \[f:\mathbb{Z}\to G,f(n):=g^n\text{ is a group homomorphism.}\]
\end{example}

\begin{proof}
    For every $m,n\in\mathbb{Z}$,
    \[f(m+n)=g^{m+n}=g^m\cdot g^n=f(m)\cdot f(n).\qedhere\]
\end{proof}

\begin{example}
    $\FZ$ is a subgroup of $(\FQ,+)$.
    $\FQ$ is a subgroup of $(\FR,+)$.
    $\FR$ is a subgroup of $(\FC,+)$.
\end{example}

\begin{example}
    $2\FZ:=\{2k\mid k\in\FZ\}$ is a subgroup of $(\FZ,+)$.
\end{example}

\begin{proof}
    For every $k,l\in\FZ$, $2k+2l=2(k+l)\in2\FZ$, and so $+$
    defines an operation on $2\FZ$.
    \begin{itemize}
        \item $+$ is associative.
        \item 
    \end{itemize}
\end{proof}


\begin{lemma}[Subgroup criterion]
    Suppose $(G,\cdot)$ is a group. A subset $H$ of $G$
    is a subgroup if it is not empty and for 
    every $x,y\in H$, $x\cdot y^{-1}\in H$.
\end{lemma}


Next we explore some of the connections between
group homomorphisms and subgroups. Suppose $(G,\cdot)$
and $(H,\cdot)$ are two groups, and $f:G\to H$
is a group homomorphism. Let $\Im(f)$ be the image
of $f$, that is $\Im(f):=\{f(g)\mid g\in G\}$,
and similar to linear algebra let the kernel of $f$ be 
\[\ker(f):=\{g\in G\mid f(g)=e_H\},\]
where $e_H$ is the neutral element of $H$.

\begin{theorem}
    Suppose $(G,\cdot)$ and $(H,\cdot)$ are groups,
    and $f:G\to H$ is a group homomorphism.
    Then $\Im(f)$ is a subgroup of $H$,
    and $\ker(f)$ is a subgroup of $G$.
\end{theorem}
\end{document}