\documentclass{article}
\usepackage{amsmath,amssymb}
\usepackage{amsthm}
\theoremstyle{plain}
\newtheorem{lemma}{Lemma}
\newtheorem{theorem}{Theorem}
\newtheorem{proposition}{Proposition}
\theoremstyle{definition}
\newtheorem*{definition}{Definition}
\newtheorem*{example}{Example}

\newcommand{\R}{\mathrm{R}}
\newcommand{\FR}{\mathbb{R}}
\newcommand{\FC}{\mathbb{C}}
\newcommand{\FQ}{\mathbb{Q}}
\newcommand{\FZ}{\mathbb{Z}}
\renewcommand{\Im}{\operatorname{Im}}

\usepackage{xcolor}

\begin{document}


\section{Introduction}


Historically algebra was created to understand zeros of
polynomial equations. Along the way the importance of various
system of numbers and their symmetries became evident.
The importance of symmetries of objects in other parts of
math and other sciences turned it into an important part of
algebra which has connections with geometry, analysis,
combinatorics, topology, etc. Symmetries of objects are studied
in group theory, which is the main subject of our course.

We start by recalling some of the basic concepts from
set theory and congruence arithmetic.

\paragraph{Equivalence Relation}
Let $X$ be a non-empty set. A relation over $X$ is a subset
$\R$ of $X\times X$. If $(x,y)\in\R$, we say $x$ is $\R$-related
to $y$ and write $x\R y$. Suppose $\R$ is a relation over $X$.
Then:
\begin{itemize}
    \item $\R$ is called reflexive if $\forall x\in X$, $x\R x$.
    \item $\R$ is called symmetric if $\forall x,y\in X$, $x\R y\Rightarrow y\R x$.
    \item $\R$ is called transitive if $\forall x,y\in X$, $x\R y,y\R z\Rightarrow x\R z$.
\end{itemize}

$\R$ is called an equivalent relation if $\R$ is reflexive, symmetric
and transitive.

Equivalent relations are essentially about equality with
respect to certain measurment. The following example
is an important indication of this concept:

\begin{example}
    Suppose $X$ and $Y$ are two non-empty sets and $f:X\to Y$
    is a function. Let $\sim$ be the following relation over $X$:
    \[\forall x_1,x_2\in X,x_1\sim x_2\Longleftrightarrow f(x_1)=f(x_2).\]
    Then $\sim$ is an equivalent relation.
\end{example}

Alternatively equivalent relations partition $X$ into subsets
that share the same property. For instance in the previous example
the shared property is having the same value under the function $f$.
Let's recall that $P$ is called a partition of a non-empty set
$X$, if 
\begin{itemize}
    \item $P$ consists of non-empty subsets of $X$, (subsets)
    \item $A,B\in P$ and $A\neq B\Rightarrow A\cap B=\varnothing$, (disjoint)
    \item $\forall x\in X$, $\exists A\in P$ such that $x\in A$. (covering)
\end{itemize}

Suppose $P$ is a partition of $X$. Then we get a classification function
from $X$ to $P$: $X\to P$, $x\mapsto [x]_P$ where $[x]_P$ is the 
unique element of $P$ which contains $x$. Notice that because of
the covering condition $x$ is contained in some element of $P$,
and because of the disjointness condition $x$ is in a unique
element of $P$. By the previous example, $x\sim y\Leftrightarrow [x]_P=[y]_P$
is an equivalent relation. So we can obtain the following lemma:

\begin{lemma}
    Suppose $P$ is a partition of a non-empty set $X$.
    For $x,y\in X$, $x\sim y$ if $x$ and $y$ are in the same element
    of $P$. Then $\sim$ is an equivalent relation.
\end{lemma}

\begin{proof}
    For $x\in X$, let $[x]_P$ be the unique element of $P$
    which contains $x$. So $x\mapsto[x]_P$ is a function from $X$
    to $P$. By the previous example, $x\sim y\Leftrightarrow[x]_P=[y]_P$
    is an equivalent relation over $X$. Notice that this means 
    $x\sim y$ exactly when $x$ and $y$ are in the same element of $P$.
\end{proof}

Starting with an equivalent relation $\sim$ over a non-empty
set $X$, we can partition $X$ with respect to $\sim$,
as we show next.

For $x\in X$, let $[x]:=\{y\in X\mid y\sim x\}$ (all the elements
that are $\sim$-related to $x$) We call $[x]$ the equivalent class
of $x$ with respect to $\sim$. When $x\sim y$, we say $x$ is equivalent to
$y$ with respect to $\sim$.

\begin{proposition}
    Suppose $\sim$ is an equivalent relation over a non-empty
    set $X$. Then $\{[x]\mid x\in X\}$ is a partition of $X$.
\end{proposition}

\begin{lemma}
    $x\sim y\Leftrightarrow [x]=[y]$.
\end{lemma}

\section{Multiplicative structure of integers mod $n$}
Here we want to investigate what elements of $\mathbb{Z}_n$
have multiplicative inverse.
\begin{definition}
    We say $[a]_n\in\mathbb{Z}_n$ has a \textcolor{blue}{multiplicative inverse}
    if $[a]_n\cdot [a']_n=[1]_n$ for some $[a']_n\in\mathbb{Z}_n$. We say $[a]_n$
    is \textcolor{blue}{a unit of $\mathbb{Z}_n$} if it has a multiplicative 
    inverse. The set of all the units of $\mathbb{Z}_n$ is denoted
    by $\mathbb{Z}_n^*$.
\end{definition}

\begin{theorem}
    Suppose $n\in\mathbb{Z}$ and $n\geq 2$. Then
    \[\mathbb{Z}_n^*=\{[a]_n\mid \gcd(a,n)=1\}.\]
    Moreover $|\mathbb{Z}_n^*|=|\{a\in\mathbb{Z}\mid 1\leq a\leq n,\gcd(a,n)=1\}|$.

    (The left hand side of the above equality is denoted
    by $\phi(n)$ and it is called \textbf{Euler's phi function}.)
\end{theorem}

\begin{proof}
    ($\subset$) Suppose $[a]_n\in\mathbb{Z}_n^*$. Then $[a]_n[a']_n=[1]_n$
    for some $a'\in\mathbb{Z}$. Hence $[aa']_n=[1]_n$ which implies
    that $aa'\equiv 1\pmod 1$. (Earlier we proved that $b\equiv b'\pmod n$
    implies $\gcd(b,n)=\gcd(b',n)$). Hence $\gcd(aa',n)=\gcd(1,n)=1$.
    Therefore $\gcd(a,n)=1$.

    ($\supset$) Suppose $\gcd(a,n)=1$. Then $1=ra+sn$
    for some $r,s\in\mathbb{Z}$. Since $ra+sn=1$, we obtain that
    \[ra\equiv 1\pmod n\]
    This implies that $[ra]_n=[1]_n$, and so 
    $[r]_n[a]_n=[1]_n$. Therefore $[a]_n\in\mathbb{Z}_n$.
\end{proof}

\section{Group}
Group theory is (mostly) about symmetrices of objects.
In some interesting examples in geometry, combinatorics,
or even chemistry, knowing the symmetrices uniquely 
determine the object. One can say that at a meta-level,
the whole mathematics (and in general sciences) is about
finding patterns as we want to reduce the amount of
data that we need to store. (Lowering the complexity
of the objects that we are studying.)

We start with an axiomatic definition of groups, and then 
give the relation with symmetrices.
\begin{definition}
    Suppose $G$ is a non-empty set and $(g_1,g_2)\mapsto g_1\cdot g_2$
    is an operator on $G$ (that means it is a function from $G\times G$
    to $G$). We say $(G,\cdot)$ (or simply $G$) is a group if the 
    following properties hold.
    \begin{itemize}
        \item (Associative) $\forall g_1,g_2,g_3\in G$, $g_1\cdot (g_2\cdot g_3)=(g_1\cdot g_2)\cdot g_3$
        \item (Neutral element) $\exists e\in G$, $\forall g\in G$, $g\cdot e=e\cdot g=g$
        \item (Inverse) $\forall g\in G$, $\exists g'\in G$, $g\cdot g'=g'\cdot g=e$, where $e$ is a neutral element.
    \end{itemize}
\end{definition}

We have already seen some exmaples of groups.
\begin{example}
    $(\mathbb{Z},+)$, $(\mathbb{Q},+)$, $(\FR,+)$ and $(\mathbb{C},+)$ are groups.
\end{example}

\begin{example}
    $(\mathbb{Q}\backslash\{0\},\cdot)$, $(\FR\backslash\{0\},\cdot)$, and 
    $(\mathbb{C}\backslash\{0\},\cdot)$ are groups.
\end{example}

\begin{example}
    For every integer $n\geq 2$, $(\mathbb{Z}_n,+)$ is a group.
\end{example}

\begin{example}
    For every integer $n\geq 2$, $(\mathbb{Z}_n^*,\cdot)$ is a group.
\end{example}

In some of the examples, we showed the uniqueness
of a neutral element when it exists. Next we show
this property in a general setting.

\begin{lemma}
    Suppose $G$ is a non-empty set, and $(g_1,g_2)\mapsto g_1\cdot g_2$
    is an operation. Suppose $e,e'\in G$ are neutral
    elements of $\cdot$. Then $e=e'$. In particular,
    in a group, there is a unique neutral element.
\end{lemma}

\begin{proof}
    Since $e$ is a neutral element, $e\cdot e'=e'$.
    Because $e'$ is a neutral element, $e\cdot e'=e$.
    Altogether we have 
    \[e'=e\cdot e'=e'.\qedhere\]
\end{proof}

Next we show the uniqueness of inverse in a group.

\begin{lemma}
    Suppose $(G,\cdot)$ is a group. Then every element
    $g$ has a unique inverse. That means if $g_1,g_2$
    are inverses of $g$, then $g_1=g_2$
\end{lemma}


\section{Homomorphism and subgroups}

Whenever we learn about a new structure in mathematics, we 
should study the functions between these objects that preserve
their properties. These functions are often called homomorphism.
(In a very vague sense homomorphisms give us a global understanding
of the objects.) Another point of view is from inside:
we often study subsets that share the same property. For
instance in linear algebra, the objects of interest are vector
spaces, the homomorphisms are linear maps, and subsets that
share the same properties are subspaces. We do the same for groups.

\begin{definition}
    Suppose $(G,\cdot)$ and $(H,*)$ are two groups.
    Then a function $f:G\to H$ is called a group homomorphism if 
    for every $g_1,g_2\in G$, $f(g_1\cdot g_2)=f(g_1)*f(g_2)$.
\end{definition}

\begin{definition}
    Suppose $(G,\cdot)$ is a group. Then a subset $K$ of $G$
    is called a subgroup of $G$ if $K$ is a group with respect to the
    operation $\cdot$.
\end{definition}

Next we see a few examples.

\begin{example}
    Suppose $n$ is an integer and $n\geq 2$. Then
    \[c_n:\mathbb{Z}\to\mathbb{Z}_n,c_n(a)=[a]_n\]
    is a group homomorphism.
\end{example}

\begin{proof}
    For $\forall a,b\in\mathbb{Z}$,
    \[c_n(a+b)=[a+b]_n=[a]_n+[b]_n=c_n(a)+c_n(b).\qedhere\]
\end{proof}

\begin{example}
    $f:\mathbb{Z}\to\mathbb{Z}$, $f(x)=-x$ is a group homomorphism.
\end{example}

\begin{proof}
    For every $x,y\in\mathbb{Z}$,
    \[f(x+y)=-(x+y)=(-x)+(-y)=f(x)+f(y).\qedhere\]
\end{proof}

\begin{example}
    Let $\FR^{>0}$ be the set of positive
    real numbers. Notice that $\FR^{>0}$ is a group
    under multiplication. Then 
    \[\ln:\FR^{>0}\to\FR\text{ is a group homomorphism.}\]
\end{example}

\begin{proof}
    For every $x,y\in\FR^{>0}$,
    \[\ln(x\cdot y)=\ln(x)+\ln(y).\qedhere\]
\end{proof}

\begin{example}
    Let $N:\FC\backslash\{0\}\to\FR^{>0}$, $N(z)=|z|$. Then 
    $N$ is a group homomorphism.
\end{example}

\begin{proof}
    For every $z\in\FC\backslash\{0\}$, $|z|\in\FR^{>0}$
    and $|z_1\cdot z_2|=|z_1|\cdot|z_2|$.
\end{proof}

\begin{example}
    Let $GL_n(\FR)$ be the set of invertible $n\times n$
    real matrices. From linear algebra we know that matrix
    multiplication is associative, product of two
    invertible $n\times n$ matrices is invertible, for
    every $a$ in $GL_n(\FR)$, $a\cdot I_n=I_n\cdot a=a$
    where $I_n$ is the identity matrix.
    So $(GL_n(\FR),\cdot)$ is a group. Let $\theta:GL_n(\FR)\to GL_n(\FR)$,
    $\theta(x)=(x^t)^{-1}$ where $x^t$ is the transpose of $x$. Then
    $\theta$ is a group homomorphism.
\end{example}

\begin{proof}
    \[\theta(x\cdot y)=((x\cdot y)^t)^{-1}=(y^t\cdot x^t)^{-1}=(x^t)^{-1}\cdot(y^t)^{-1}=\theta(x)\cdot\theta(y).\qedhere\]
\end{proof}

\begin{example}
    Suppose $(G,\cdot)$ is a group. Then $f:G\to G$, $f(g)=g^{-1}$
    is a group homomorphism if and only if $G$ is abelian.
\end{example}

\begin{proof}
    ($\Rightarrow$) For every $g,h\in G$, $f(g\cdot h)=f(g)\cdot f(h)$.
    Then 
    \begin{equation}
        (g\cdot h)^{-1}=g^{-1}\cdot h^{-1}\Rightarrow h^{-1}\cdot g^{-1}=g^{-1}\cdot h^{-1}\tag{(I)}
    \end{equation}
    For $x,y\in G$, let $g=x^{-1}$ and $h=y^{-1}$ in (I). Then
    we obtain $(y^{-1})^{-1}\cdot (x^{-1})^{-1}=(x^{-1})^{-1}\cdot(y^{-1})^{-1}$.\
    Since $(x^{-1})^{-1}=x$ and $(y^{-1})^{-1}=y$, we conclude
    $y\cdot x=x\cdot y$. Therefore $G$ is abelian.

    ($\Leftarrow$) $f(g\cdot h)=(g\cdot h)^{-1}=h^{-1}\cdot g^{-1}=f(h)\cdot f(g)=f(g)\cdot f(h)$.
\end{proof}

\begin{example}
    Suppose $(G,\cdot)$ is a group and $g\in G$. Let
    \[c_g:G\to G,c_g(x):=g\cdot x\cdot g^{-1}.\]
    Then $c_g$ is a group homomorphism.
\end{example}

\begin{proof}
    For $x,y\in G$, we have to show that 
    \[c_g(x\cdot y)=c_g(x)\cdot c_g(y).\]
    We have $c_g(x\cdot y)=g\cdot x\cdot y\cdot g^{-1}$ and 
    \[\begin{split}
        c_g(x)\cdot c_g(y)
        &=(g\cdot x\cdot g^{-1})\cdot(g\cdot y\cdot g^{-1})\\
        &=g\cdot x\cdot (g^{-1}\cdot g)\cdot y\cdot g^{-1}\\
        &=g\cdot x\cdot e_G\cdot y\cdot g^{-1}\\
        &=g\cdot x\cdot y\cdot g^{-1}
    \end{split}\]
    Therefore $c_g(x\cdot y)=c_g(x)\cdot c_g(y)$.
\end{proof}

\begin{example}
    Suppose $(G,\cdot)$ is a group and $g\in G$. Then
    \[f:\mathbb{Z}\to G,f(n):=g^n\text{ is a group homomorphism.}\]
\end{example}

\begin{proof}
    For every $m,n\in\mathbb{Z}$,
    \[f(m+n)=g^{m+n}=g^m\cdot g^n=f(m)\cdot f(n).\qedhere\]
\end{proof}

\begin{example}
    $\FZ$ is a subgroup of $(\FQ,+)$.
    $\FQ$ is a subgroup of $(\FR,+)$.
    $\FR$ is a subgroup of $(\FC,+)$.
\end{example}

\begin{example}
    $2\FZ:=\{2k\mid k\in\FZ\}$ is a subgroup of $(\FZ,+)$.
\end{example}

\begin{proof}
    For every $k,l\in\FZ$, $2k+2l=2(k+l)\in2\FZ$, and so $+$
    defines an operation on $2\FZ$.
    \begin{itemize}
        \item $+$ is associative.
        \item 
    \end{itemize}
\end{proof}


\begin{lemma}[Subgroup criterion]
    Suppose $(G,\cdot)$ is a group. A subset $H$ of $G$
    is a subgroup if it is not empty and for 
    every $x,y\in H$, $x\cdot y^{-1}\in H$.
\end{lemma}


Next we explore some of the connections between
group homomorphisms and subgroups. Suppose $(G,\cdot)$
and $(H,\cdot)$ are two groups, and $f:G\to H$
is a group homomorphism. Let $\Im(f)$ be the image
of $f$, that is $\Im(f):=\{f(g)\mid g\in G\}$,
and similar to linear algebra let the kernel of $f$ be 
\[\ker(f):=\{g\in G\mid f(g)=e_H\},\]
where $e_H$ is the neutral element of $H$.

\begin{theorem}
    Suppose $(G,\cdot)$ and $(H,\cdot)$ are groups,
    and $f:G\to H$ is a group homomorphism.
    Then $\Im(f)$ is a subgroup of $H$,
    and $\ker(f)$ is a subgroup of $G$.
\end{theorem}
\end{document}